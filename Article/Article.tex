% Options for packages loaded elsewhere
\PassOptionsToPackage{unicode}{hyperref}
\PassOptionsToPackage{hyphens}{url}
%
\documentclass[
  11pt,
]{article}
\usepackage[]{mathpazo}
\usepackage{amssymb,amsmath}
\usepackage{ifxetex,ifluatex}
\ifnum 0\ifxetex 1\fi\ifluatex 1\fi=0 % if pdftex
  \usepackage[T1]{fontenc}
  \usepackage[utf8]{inputenc}
  \usepackage{textcomp} % provide euro and other symbols
\else % if luatex or xetex
  \usepackage{unicode-math}
  \defaultfontfeatures{Scale=MatchLowercase}
  \defaultfontfeatures[\rmfamily]{Ligatures=TeX,Scale=1}
\fi
% Use upquote if available, for straight quotes in verbatim environments
\IfFileExists{upquote.sty}{\usepackage{upquote}}{}
\IfFileExists{microtype.sty}{% use microtype if available
  \usepackage[]{microtype}
  \UseMicrotypeSet[protrusion]{basicmath} % disable protrusion for tt fonts
}{}
\makeatletter
\@ifundefined{KOMAClassName}{% if non-KOMA class
  \IfFileExists{parskip.sty}{%
    \usepackage{parskip}
  }{% else
    \setlength{\parindent}{0pt}
    \setlength{\parskip}{6pt plus 2pt minus 1pt}}
}{% if KOMA class
  \KOMAoptions{parskip=half}}
\makeatother
\usepackage{xcolor}
\IfFileExists{xurl.sty}{\usepackage{xurl}}{} % add URL line breaks if available
\IfFileExists{bookmark.sty}{\usepackage{bookmark}}{\usepackage{hyperref}}
\hypersetup{
  pdftitle={Reproduce the results of the article ``Relationship of gender differences in preferences to economic development and gender equality''},
  pdfauthor={Sara Cerioli; Andrey Formozov},
  pdfkeywords={pandoc, r markdown, knitr},
  hidelinks,
  pdfcreator={LaTeX via pandoc}}
\urlstyle{same} % disable monospaced font for URLs
\usepackage[margin=1in]{geometry}
\usepackage{longtable,booktabs}
% Correct order of tables after \paragraph or \subparagraph
\usepackage{etoolbox}
\makeatletter
\patchcmd\longtable{\par}{\if@noskipsec\mbox{}\fi\par}{}{}
\makeatother
% Allow footnotes in longtable head/foot
\IfFileExists{footnotehyper.sty}{\usepackage{footnotehyper}}{\usepackage{footnote}}
\makesavenoteenv{longtable}
\usepackage{graphicx,grffile}
\makeatletter
\def\maxwidth{\ifdim\Gin@nat@width>\linewidth\linewidth\else\Gin@nat@width\fi}
\def\maxheight{\ifdim\Gin@nat@height>\textheight\textheight\else\Gin@nat@height\fi}
\makeatother
% Scale images if necessary, so that they will not overflow the page
% margins by default, and it is still possible to overwrite the defaults
% using explicit options in \includegraphics[width, height, ...]{}
\setkeys{Gin}{width=\maxwidth,height=\maxheight,keepaspectratio}
% Set default figure placement to htbp
\makeatletter
\def\fps@figure{htbp}
\makeatother
\setlength{\emergencystretch}{3em} % prevent overfull lines
\providecommand{\tightlist}{%
  \setlength{\itemsep}{0pt}\setlength{\parskip}{0pt}}
\setcounter{secnumdepth}{-\maxdimen} % remove section numbering
\usepackage[]{biblatex}
\addbibresource{bibliography.bibtex}

\title{\textbf{Reproduce the results of the article ``Relationship of gender
differences in preferences to economic development and gender
equality''}}
\author{Sara Cerioli \and Andrey Formozov}
\date{\texttt{r\ format(Sys.time(),\ \textquotesingle{}\%B\ \%d,\ \%Y\textquotesingle{})}}

\begin{document}
\maketitle
\begin{abstract}
This study attempts to reproduce the results of the article of
\textcite{FH} measuring the gender differences economic preferences
relating them to economic development and to gender equality of the
countries. In the original paper, the authors use data from the Gallup
World Poll 2012, which included a Global Preference Survey conducted on
almost 80000 people in 76 countries all around the world. The dataset
covers almost 90\% of the world population representation, with each
country having around 1000 participants answering questions related to
their time preference (patience), altruism, will of risk taking,
negative and positive reciprocity, and trust. The dataset is available
in its integrity only with a license to be paid. The free version has
only partial data that can also be used for this purpose because,
according to the FH study, the gender differences can be studied also
only taking in consideration a smaller number of predictors (according
to the supplementary material, see \autocite{FH_SM}). In this
replication study, therefore, we use only a subset of predictors that
are made publicly available to check whether the results can still be
reproduced and are consistent. The outcome of the replication is that we
see similar results as the ones obtained by the original authors for the
relationship of gender differences and the economic development, but
with differences (some times minor, some times significantly large)
regarding the gender equality, especially when comparing the results of
the single indexes building the general Gender Equality Index. Those
differences seem to arise from different handling of the country-level
variables, but a further check couldn't be done because the data used in
the original article are not provided to us.
\end{abstract}

\hypertarget{highlights}{%
\section{Highlights}\label{highlights}}

Code and data used for this replication study (according to licenses)
are available under the git repository
\url{https://github.com/scerioli/Global-Preferences-Survey}.

\hypertarget{introduction}{%
\section{1. Introduction}\label{introduction}}

Gender differences are nowadays extensively used as arguments and
counter-arguments for decision and policy making, and the differences
concerning the economic behaviors, such as happiness \autocite{SPSU},
competition \autocite{CG} and \autocite{GLL}, or work preferences
\autocite{BG}, are being studied in many sectors of the economy and
economy-related fields.

One of the problems common for many experiments in social sciences is
the lack of large and heterogeneous datasets that can be used to check
for such differences reducing some of the bias induced, for example, by
having students or specific sets of people interviewed for the study.

The Gallup World Poll 2012 included a Global Preference Survey conducted
on almost 80000 people in 76 countries all around the world, that aimed
to fill this gap: Covering almost 90\% of the world population
representation, with each country having around 1000 participants
answering questions related to their time preference (patience),
altruism, will of risk taking, negative and positive reciprocity, and
trust.

The dataset provides a unique insight in the economic preferences of a
heterogeneous amount of people. The original study published in the
Quarterly Journal of Economics \autocite[133 (4)
pp.~1645-1692]{QJE_Falk} focused on more general questions about the
economic preferences distributions in different countries, trying to
explore different covariates from the Gallup World Poll. While, the
subsequent article, replicated in this work, focused specifically on the
gender differences arising from the previous study.

The main question that the article wants to study is whether the gender
differences in economic preferences increase or decrease as the economic
development and gender equality of the countries increase. In the first
scenario, the gender differences increase as the economic development
increases because the gender-neutral goal of substistence is removed,
and therefore the real preferences can be pursued. Moreover, since those
countries are usually also the ones with more gender equal societies, we
would have more women and men allowed to express their desires and
preferences in an independent way. This would be the so-called resources
hypothesis. On the other hand, there is the social role hypothesis
stating that the more economically developed and gender-equal the
country, the smaller the gender differences because of the attenuation
of the social roles related to the genders. The conclusion of the
article is that the trends in the data shows a positive correlation of
gender differences with GDP p/c and with the gender equality of the
countries, and thus ``confirming'' the resources hypothesis.

The motivation for the replication study came from the wish to apply
different analysis beyond the OLS, as for instance multilevel cumulative
link models \emph{{[}cit. link{]}}. In order to achieve that, one needs
first to replicate the original analysis. The replication allowed us to
question the data sources and the methods used for data cleaning, and
even if the data is not complete, we think it is still interesting to
replicate such studies to validate the procedures for the sake of good
science.

\hypertarget{methods}{%
\section{2. Methods}\label{methods}}

\hypertarget{overview}{%
\subsection{Overview}\label{overview}}

We replicate the results using the R programming language version 4.0.3
(2020-10-10), and its open-source IDE RStudio for an easy access of the
code. The following packages with respective version are used:

\begin{longtable}[]{@{}lll@{}}
\toprule
Package & \(\quad\) & Version\tabularnewline
\midrule
\endhead
data.table & & 1.13.2\tabularnewline
bit64 & & 4.0.5\tabularnewline
bit & & 4.0.4\tabularnewline
plyr & & 1.8.6\tabularnewline
dplyr & & 1.0.5\tabularnewline
haven & & 2.4.2\tabularnewline
ggplot2 & & 3.3.2\tabularnewline
missMDA & & 1.18\tabularnewline
\bottomrule
\end{longtable}

\hypertarget{data-collection-cleaning-and-standardization}{%
\subsection{Data Collection, Cleaning, and
Standardization}\label{data-collection-cleaning-and-standardization}}

The data used by the authors is not fully available because of two
reasons:

\begin{enumerate}
\def\labelenumi{\arabic{enumi}.}
\item
  \textbf{Data paywall:} Some sociodemographic variables (for instance,
  education level or income quintile) are not part of the Global
  Preference Survey, but of the Gallup World Poll dataset. Check the
  website of the
  \href{https://www.briq-institute.org/global-preferences/home}{briq -
  Institute on Behavior \& Inequality} for more information on it.
\item
  \textbf{Data used in study is not available online:} This is what
  happened for the log GDP p/c calculated in 2005 US dollars (which is
  not directly available online). We decided to calculate the log GDP
  p/c in 2010 US dollars because it was easily available, which should
  not change the main findings of the article.
\end{enumerate}

An additional issue that we faced while trying to reproduce the results
of the article has been the missing data. We will treat this specific
issue later on because it requires a bit of background.

The procedure for cleaning is described for each dataset in the
corresponding section below. After manually cleaning the dataset, we
standardized the names of the countries and merged the datasets into
one.

\hypertarget{global-preferences-survey}{%
\subsubsection{Global Preferences
Survey}\label{global-preferences-survey}}

This data is protected by copyright and can't be given to third parties.

To download the GPS dataset, go to the website of the Global Preferences
Survey in the section ``downloads''. There, choose the ``Dataset'' form
and after filling it, we can download the dataset.

\hypertarget{gdp-per-capita}{%
\subsubsection{GDP per capita}\label{gdp-per-capita}}

From the \href{https://data.worldbank.org/indicator/}{website of the
World Bank}, one can access the data about the GDP per capita on a
certain set of years. We took the GDP per capita (constant 2010 US\$),
made an average of the data from 2003 until 2012 for all the available
countries, and matched the names of the countries with the ones from the
GPS dataset.

\hypertarget{gender-equality-index}{%
\subsubsection{Gender Equality Index}\label{gender-equality-index}}

The Gender Equality Index is composed of four main datasets. Here below
we describe where to get them (as originally sourced by the authors) and
how we treated the data within them, if needed.

\begin{itemize}
\item
  \textbf{Time since women's suffrage:} Taken from the
  \href{http://www.ipu.org/wmn-e/suffrage.htm\#Note1}{Inter-Parliamentary
  Union Website}. We prepared the data in the following way. For several
  countries more than one date where provided (for example, the right to
  be elected and the right to vote). We use the last date when both vote
  and stand for election right were granted, with no other restrictions
  commented. Some countries were colonies or within union of the
  countries (for instance, Kazakhstan in Soviet Union). For these
  countries, the rights to vote and be elected might be technically
  granted two times within union and as independent state. In this case
  we kept the first date. It was difficult to decide on South Africa
  because its history shows the racism part very entangled with women's
  rights {[}citation{]}. We kept the latest date when also Black women
  could vote. For Nigeria, considered the distinctions between North and
  South, we decided to keep only the North data because, again, it was
  showing the completeness of the country and it was the last date.
  Note: USA data doesn't take into account that also up to 1964 black
  women couldn't vote (in general, Blacks couldn't vote up to that
  year). We didn't keep this date, because it was not explicitly
  mentioned in the original dataset. This can be seen as in contrast
  with other choices made though.
\item
  \textbf{UN Gender Inequality Index:} Taken from the
  \href{http://hdr.undp.org/sites/default/files/hdr_2016_statistical_annex.pdf}{Human
  Development Report 2015}. We kept only the table called ``Gender
  Inequality Index''.
\item
  \textbf{WEF Global Gender Gap:} WEF Global Gender Gap Index Taken from
  the \href{http://reports.weforum.org/}{World Economic Forum Global
  Gender Gap Report 2015}. For countries where data was missing, data
  was added from the World Economic Forum Global Gender Gap Report 2006.
  NOTE: We modified some of the country names directly on the csv file,
  that is why we provide this as an input file.
\item
  \textbf{Ratio of female and male labour force participation:} Average
  International Labour Organization estimates from 2003 to 2012 taken
  from the World Bank database
  (\url{http://data.worldbank.org/indicator/SL.TLF.CACT.FM.ZS}). Values
  were inverted to create an index of equality. We took the average for
  the period between 2004 and 2013.
\end{itemize}

\hypertarget{missing-data-and-imputation}{%
\subsubsection{Missing Data and
Imputation}\label{missing-data-and-imputation}}

During the reproduction of the article, we found that the authors didn't
write in details how they handled missing data in the indicators.

They mention on page 14 of the Supplementary Material, that (quoting):
``For countries where data were missing data were added from the World
Economic Forum Global Gender Gap Report 2006
(\url{http://www3.weforum.org/docs/WEF_GenderGap_Report_2006.pdf}).''

However, there are two problems here:

\begin{itemize}
\item
  Regarding the year when women received the right to vote in a specific
  country. The missing values are the ones coming from the United Arab
  Emirates and Saudi Arabia, that neither in 2006 (when the WEF Global
  Gender Gap Report that the authors quote as a reference for the
  missing values) nor now (in 2021) have guaranteed the right to vote
  for women yet.
\item
  There are missing data also in the other sources that the authors
  quote. So a quick search for the missing countries of the WEF report
  of 2015, shows us that these countries can't be found in the report of
  2006 either.
\end{itemize}

These two unclear points, even though in our understanding not crucial
for the replication of the analysis, are not desirable, but couldn't be
further clarified with the authors.

The problem of missing data for a given countries often does not
influence much the overall trends of found correlations, however, very
relevant if one would to see the implications of the study with respect
to one certain country of interest.

\hypertarget{data-analysis}{%
\subsection{Data Analysis}\label{data-analysis}}

The article uses several methods commonly accepted in the field:

\begin{itemize}
\item
  Linear regression for each Country for each preference to extract the
  gender coefficient as a measure of the gender differences.
\item
  Principal Component Analysis on 6 gender coefficients to summarize an
  overall measure of the gender differences, and 4 gender equality
  indexes of the countries to summarize an overall Gender Equality
  Index.
\item
  Variable Conditioning to separate further between economic development
  and gender equality in the country.
\end{itemize}

\hypertarget{linear-model-on-each-country-for-each-preference}{%
\subsubsection{Linear Model on Each Country for Each
Preference}\label{linear-model-on-each-country-for-each-preference}}

Starting from the complete dataset (meaning with removed NA rows), we
wanted to reproduce the data plotted in Fig. S2. regarding the gender
differences and economic development by preference and by country.

As already mentioned in the previous paragraph, part of the data to
reproduce the article is under restricted access: education level and
household income quintile on the individual level are not available in
open access version.

As the FH article addresses the gender differences, the main focus is on
that individual variable and all the others provided in the dataset
(education level, income quitile, age, and subjective math skills) are
taken as control variables, meaning that the presence of these variables
may not affect the result of the correlation.

In \textcite{FH_SM} they check that the role of these variables to any
extent negligible in the overall correlation and we therefore decided to
continue the analysis without using the two variables education level
and household income quintile for building the linear models.

Note that the code can be easily modified to include this variable once
the full dataset is available.

The linear model for each country is created using the equation:

\(p_i = \beta_1^c female_i + \beta_2^c age_i + \beta_3^c age^2_i + \beta_4^c subjective \ math \ skills_i + \epsilon_i\)

This resulted in 6 models -- one for each preference measure, \(p_i\) --
having intercept and 4 weights, each of the weight being related to the
variable in the formula above. The weight for the dummy variable
``female'', \(\beta_1^c\), is used as a measure of the country-level
gender difference. Therefore, in total, we have 6 weights that represent
the preference difference related to the gender for 76 countries.

\hypertarget{principal-component-analysis}{%
\subsubsection{Principal Component
Analysis}\label{principal-component-analysis}}

To summarise the average gender difference among the six economic
preferences, we performed a principal component analysis (PCA) on the
gender coefficients from the linear models. The PCA is a dimensionality
reduction technique which allows to ``reshape'' the 6 coefficients into
other mixed components that maximise the variance. The first component
of the PCA has then been used as a summary index of average gender
differences in preferences.

We performed a PCA also on the four datasets used for Gender Equality,
to extract a summarised Gender Equality Index \emph{(more on the
structure of the constructed Equality Index in section --\textgreater{}
do we want to have it in the supplementary material together with the
critics? Not sure)}.

\hypertarget{variable-conditioning}{%
\subsubsection{Variable Conditioning}\label{variable-conditioning}}

To separate the effects of the economic development and the gender
equality, a conditional analysis was performed (\textcite{FW}, and
\textcite{Lovell}). To generalise, if one wants to estimate the
correlation of x and y conditioning on z, one needs to perform a double
linear regression:

\begin{itemize}
\item
  First, regressing x on z and extracting the residuals
\item
  Second, regressing y on z, and extracting the residuals.
\end{itemize}

In the end, one needs to take the so calculated residuals of x on z and
of y on z, and make a last regression to calculate the correlation.

In practice, if we are interested in checking the influence of the
economic development on the summarised gender differences, conditioning
on the gender equality, we would need to regress the economic
development on the gender equality index, then the average gender
differences regressed on the gender equality index, and finally regress
the residuals of the average gender differences on the residuals of the
economic development.

\hypertarget{comparison-to-the-original-article}{%
\section{3. Comparison to the Original
Article}\label{comparison-to-the-original-article}}

In this section, we describe how to reproduce the plots and compare the
results in terms of z-scores.

\hypertarget{reproducing-the-plots-of-the-main-article}{%
\subsection{Reproducing the Plots of the Main
Article}\label{reproducing-the-plots-of-the-main-article}}

To reproduce the plot of Fig. 1A, we grouped the countries in quartiles
based on the logarithm of their average GDP p/c, extracted the mean of
each preference from the gender coefficients (the \(\beta_1^c\)) of the
countries for each quartile, after standardizing them. The same method
was applied to the Gender Equality Index in correlation to the gender
differences for each economic preference, to reproduce the plot in Fig.
1C.

Then, we related the magnitude of the summarised gender difference
coefficients (the first component of the PCA) with the logarithm of the
average GDP per capita to see the effect of the economic development.
This reproduced Fig. 1B of the original article. We used a linear model
to fit the correlation and extract the p-value, and for the plot the
variables on the y-axis were additionally transformed as
(y-y\_min)/(y\_max-y\_min). We applied the same method to extract the
correlation between the Gender Equality Index and the summarised gender
preference, to see the effect of the gender equality in the countries
(Fig. 1D). Note that here also the Gender Equality Index is transformed
to be on a scale between 0 and 1.

We finally reproduced the plots in Fig. 2A-F using the variable
conditioning analysis. This has been done for the economic development,
for the Gender Equality Index, and for each of the four indicators
building the Gender Equality Index. The variable used on the y-axis is
the first Principal Component of the PCA made on the gender differences
on the six preferences. All the variables used have been standardize to
have mean at 0 and standard deviation of 1 before applying the
conditional analysis. Using the residuals, built as described in the
Data Analysis section of the Method paragraph, we performed a linear
regression on the data points, and we extracted correlation coefficients
and p-values.

\hypertarget{tables-and-z-scores}{%
\subsection{Tables and z-scores}\label{tables-and-z-scores}}

The comparison to the original article has been done using the z-scores
on the correlation coefficients, and checking if the statistical
significance was at the same level. \emph{What about the slope
coefficients?}

Here below we report the tables with the corresponding values of the
correlation for the original article, our replication study, and the
z-scores calculated from them. The sample size of the data, needed for
the calculation of the z-scores, is always 76 (the number of the
countries involved in this studies), except when using the Gender
Equality Index, where due to the missing data, the number of countries
in the sample was reduced to 68. \emph{{[}Check here the sample size of
the FH and the number of countries used in the other gender equality
indeces!{]}} We also indicate the significance level for each
correlation using the following scheme:

Signifincance \(\le\) 0.001 (***), \(\le\) 0.01 (**), \(\le\) 0.05 (*)

\hypertarget{table-1-correlations-between-log-gdp-pc-and-country-level-gender-differences}{%
\paragraph{Table 1: Correlations between Log GDP p/c and country-level
gender
differences}\label{table-1-correlations-between-log-gdp-pc-and-country-level-gender-differences}}

\begin{longtable}[]{@{}llll@{}}
\toprule
Variable & Corr. original article & Corr. this analysis &
z-score\tabularnewline
\midrule
\endhead
Altruism & 0.58*** & 0.6*** & -0.19\tabularnewline
Trust & 0.59*** & 0.53*** & 0.53\tabularnewline
Positive Reciprocity & 0.31*** & 0.29* & 0.13\tabularnewline
Positive Reciprocity & 0.35*** & 0.42*** & -0.50\tabularnewline
Risk Taking & 0.37*** & 0.33*** & 0.28\tabularnewline
Patience & 0.38*** & 0.49*** & -0.82\tabularnewline
\bottomrule
\end{longtable}

\hypertarget{table-2-correlations-between-gender-equality-index-and-country-level-gender-differences}{%
\paragraph{Table 2: Correlations between Gender Equality Index and
country-level gender
differences}\label{table-2-correlations-between-gender-equality-index-and-country-level-gender-differences}}

\begin{longtable}[]{@{}llll@{}}
\toprule
Variable & Corr. original article & Corr. this analysis &
z-score\tabularnewline
\midrule
\endhead
Altruism & 0.51*** & 0.49*** & 0.16\tabularnewline
Trust & 0.41*** & 0.47*** & -0.44\tabularnewline
Positive Reciprocity & 0.13 & 0.20 & -0.42\tabularnewline
Positive Reciprocity & 0.40*** & 0.30** & 0.67\tabularnewline
Risk Taking & 0.34*** & 0.26* & 0.56\tabularnewline
Patience & 0.43*** & 0.48*** & -0.37\tabularnewline
\bottomrule
\end{longtable}

\hypertarget{table-3-correlation-between-log-gdp-pc-and-gender-equality-index-and-summarised-gender-differences}{%
\paragraph{Table 3: Correlation between Log GDP p/c and Gender Equality
Index, and summarised gender
differences}\label{table-3-correlation-between-log-gdp-pc-and-gender-equality-index-and-summarised-gender-differences}}

\begin{longtable}[]{@{}llll@{}}
\toprule
Variable & Corr. original article & Corr. this analysis &
z-score\tabularnewline
\midrule
\endhead
Log GDP p/c & 0.6685*** & 0.7119*** & -0.46\tabularnewline
Gender Equality Index & 0.5580*** & 0.5852*** & -0.26\tabularnewline
\bottomrule
\end{longtable}

\hypertarget{table-4-conditional-analysis-to-separate-the-impacts-of-economic-development-and-gender-equality-on-gender-differences}{%
\paragraph{Table 4: Conditional analysis to separate the impacts of
economic development and gender equality on gender
differences}\label{table-4-conditional-analysis-to-separate-the-impacts-of-economic-development-and-gender-equality-on-gender-differences}}

\begin{longtable}[]{@{}llll@{}}
\toprule
\begin{minipage}[b]{0.22\columnwidth}\raggedright
Variable\strut
\end{minipage} & \begin{minipage}[b]{0.22\columnwidth}\raggedright
Residualized on\strut
\end{minipage} & \begin{minipage}[b]{0.22\columnwidth}\raggedright
Slope coeff. original article\strut
\end{minipage} & \begin{minipage}[b]{0.22\columnwidth}\raggedright
Slope coeff. this analysis\strut
\end{minipage}\tabularnewline
\midrule
\endhead
\begin{minipage}[t]{0.22\columnwidth}\raggedright
Log GDP p/c\strut
\end{minipage} & \begin{minipage}[t]{0.22\columnwidth}\raggedright
Gender Equality Index\strut
\end{minipage} & \begin{minipage}[t]{0.22\columnwidth}\raggedright
0.5258***\strut
\end{minipage} & \begin{minipage}[t]{0.22\columnwidth}\raggedright
0.5695***\strut
\end{minipage}\tabularnewline
\begin{minipage}[t]{0.22\columnwidth}\raggedright
Gender Equality Index\strut
\end{minipage} & \begin{minipage}[t]{0.22\columnwidth}\raggedright
Log GDP p/c\strut
\end{minipage} & \begin{minipage}[t]{0.22\columnwidth}\raggedright
0.3192***\strut
\end{minipage} & \begin{minipage}[t]{0.22\columnwidth}\raggedright
0.2777*\strut
\end{minipage}\tabularnewline
\begin{minipage}[t]{0.22\columnwidth}\raggedright
WEF Global Gender Gap\strut
\end{minipage} & \begin{minipage}[t]{0.22\columnwidth}\raggedright
Log GDP p/c\strut
\end{minipage} & \begin{minipage}[t]{0.22\columnwidth}\raggedright
0.2327***\strut
\end{minipage} & \begin{minipage}[t]{0.22\columnwidth}\raggedright
0.2014*\strut
\end{minipage}\tabularnewline
\begin{minipage}[t]{0.22\columnwidth}\raggedright
UN Gender Equality Index\strut
\end{minipage} & \begin{minipage}[t]{0.22\columnwidth}\raggedright
Log GDP p/c\strut
\end{minipage} & \begin{minipage}[t]{0.22\columnwidth}\raggedright
0.2911\strut
\end{minipage} & \begin{minipage}[t]{0.22\columnwidth}\raggedright
0.2355\strut
\end{minipage}\tabularnewline
\begin{minipage}[t]{0.22\columnwidth}\raggedright
F/M in Labor Force Participation\strut
\end{minipage} & \begin{minipage}[t]{0.22\columnwidth}\raggedright
Log GDP p/c\strut
\end{minipage} & \begin{minipage}[t]{0.22\columnwidth}\raggedright
0.2453*\strut
\end{minipage} & \begin{minipage}[t]{0.22\columnwidth}\raggedright
0.1648\strut
\end{minipage}\tabularnewline
\begin{minipage}[t]{0.22\columnwidth}\raggedright
Years since Women Suffrage\strut
\end{minipage} & \begin{minipage}[t]{0.22\columnwidth}\raggedright
Log GDP p/c\strut
\end{minipage} & \begin{minipage}[t]{0.22\columnwidth}\raggedright
0.2988**\strut
\end{minipage} & \begin{minipage}[t]{0.22\columnwidth}\raggedright
0.1484\strut
\end{minipage}\tabularnewline
\bottomrule
\end{longtable}

\hypertarget{discussion-of-the-results}{%
\section{4. Discussion of the Results}\label{discussion-of-the-results}}

Comparing the results of our analysis to the one from the original
paper, starting with the single preferences correlations to the economic
development, we see that our analysis brings us to very similar results
in terms of correlation coefficients (see Table 1). The p-values,
although different, are all indicating a statistically significant
correlation, as in the original paper, and when calculating the z-scores
thanks to Fisher's r to z transformation, we see that each one is below
2 (which is usually taken as threshold to be statistically significant).
This means that our correlations were not statistically significantly
different from the ones in the original article.

We do the same for the Gender Equality Index (Table 2), and again we
don't find any large difference in the correlation. The p-values tend to
be different with respect to the original article, but all of them are
pointing towards the same direction. Also the z-scores are in absolute
values below 2.

In Table 3, we compare the two core concepts of the article, where the
summarised gender differences are regressed on the log GDP p/c and on
the Gender Equality Index. The correlations found are similarly
positive, strong, and statistically significant. Again, the correlations
found in our analysis are not statistically significantly different from
the correlations found in the original article.

Lastly, we have the conditional analysis (Table 4). For the two main
country-level variable, we see that the values tend to agree and be on
the same direction (similar slope coefficients and significant p-value).
But when we start to check for the single indexes, we see that there are
some differences which are worthy to discuss.

The first thing to say is that we had to make choices on how to impute
data and also how to handle the missing data (see discussion above in
paragraph ``Methods''). The main imputation on missing data has been
done on the ``time since women's suffrage'' dataset, that is where we
see a substantial difference in the results. Other datasets, on the
other hand, has not been treated for missing data but still they present
some difference. For instance, the dataset ``F/M in Labor Force
Participation'' in our analysis has a non-statistically significant
correlation, while in the original paper they found a correlation with
p-value less than 0.05.

A first thought was that this might be the result of using a different
dataset for the GDP (the 2010 USD instead of 2005), but in our opinion
this can't be an explanation but rather a check about how robust the
results are. So this question about the differences that were found is
kept open.

\hypertarget{conclusions}{%
\section{5. Conclusions}\label{conclusions}}

The study indicates that higher economic development and higher gender
equality are associated with an increase in the gender differences in
preferences, and therefore rules out the social-role theory over the
post-materialistic one: When more resources are available to both men
and women, the expression of the gender specific preferences can be
seen. Our replication leads to the same conclusions, but we have some
open questions regarding unexplained differences that might lead to
further checks on the results' robustness.

\printbibliography[title=References]

\end{document}
